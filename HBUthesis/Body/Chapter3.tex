\chapter{论文结构及要求}
毕业论文(设计)一般由以下部分组成:\cite{1}
\subitem 毕业论文(设计)封面;
\subitem 诚信承诺书;
\subitem 中英文摘要;
\subitem 目录;
\subitem 正文;
\subitem 参考文献;
\subitem 附录;
\subitem 致谢;
\subitem 成绩评定表;
\subitem 答辩记录;
\subitem 论文相似性鉴定报告 首页。
\section{毕业论文(设计)封面}

论文一律用学院统一封面(可从学院网站“资料下载”处下载)。

\section{诚信承诺书}

由学生本人亲自用黑色水笔手写签订《淮北师范大学本科生毕业论文(设计)诚信承诺 书》。诚信承诺书内容见校教字[2013]67 号文的附件 5。

\section{中英文摘要及关键词}

摘要是论文内容的简要陈述,应尽量反映论文的主要信息,内容包括研究目的、方法、 成果、结论及主要创新之处等,不含图表,不加注释,具有独立性和完整性。中文摘要不少 于 200 字,英文摘要应与中文摘相对应,且中文摘要在前,英文摘要在后。中文摘要和英文 摘要均要另起一页。

关键词是反映毕业设计(论文)主题内容的名词,是供检索使用的。中英文摘要均要有 关键词,关键词一般为 3-5 个,各关键词用分号隔开。关键词排在摘要正文部分下方。

中文摘要页中,论文主标题样式用二号、黑体、居中,单倍行距,段前 12 磅,段后 6 磅;若有子标题,子标题样式用四号、宋体、加粗,右对齐,右侧缩进 4 字符,单倍行距, 段前 0 磅,段后 24 磅。论文标题排在摘要之前。

中文摘要标题“摘要”和“关键词”几个字样式用小四、黑体、加粗、顶格,中文摘要 正文样式用小四、宋体、行距固定值 20 磅(注:行间距 12 磅为 1 行,下同)。摘要标题后 紧接写摘要正文(不另起一行)。中文关键词样式用小四、宋体、行距固定值 20 磅。

英文摘要页中,论文主标题样式用二号、Arial、加粗、居中,单倍行距,段前 12 磅, 段后 6 磅;若有子标题,子标题样式用四号、Arial、加粗,右对齐,右侧缩进 4 字符,单 倍行距,段前 0 磅,段后 24 磅。主标题和子标题的第一个词和所有实词首字母大写,虚词 首字母小写。

英文摘要标题“Abstract”和“Key words”几个字样式用小四、Times New Roman、加 粗、顶格,英文摘要正文样式用小四、Times New Roman,两端对齐,行距固定值 20 磅。每 个英文关键词首字母要大写,样式用小四、Times New Roman,行距固定值 20 磅。

\section{目录}
目录按三级标题编写,要求层次清晰,且要与正文标题一致。主要包括绪论(或引言)、 正文主体、结论(或总结)、参考文献、附录及致谢等。“目录”二字样式用小三号、黑体、 加粗、居中,单倍行距,段前 0 磅,段后 12 磅,“目”与“录”之间空四格。目录内容中文 样式用宋体,英文样式用 Times New Roman。字号、行间距自行统一样式。目录要另起一页。

\section{正文}
统一格式是保证文章结构清晰、纲目分明的重要编辑手段,正文格式要求如下:

1(空一格)一级标题 (小三、黑体、加粗、顶格,单倍行距,段前 6 磅,段后 6 磅)。

1.1(空一格)二级标题 (四号、黑体、加粗、顶格,行距固定值 20 磅,段前 6 磅,段后 6 磅)。

1.1.1(空一格)三级标题 (小四、黑体、加粗、顶格,行距固定值 20 磅,段前 6 磅,段后 6 磅)。

标题“绪论(或引言)”和“结论(或总结)”样式用小三、黑体、加粗、顶格,单倍行 距,段前 6 磅,段后 6 磅,标题不编序号。

正文内容行距固定值 20 磅。中文样式用小四、宋体;英文用小四、Times New Roman。 段落首行缩进 2 个字符。

本科毕业论文(设计)正文字数不少于 5000 字。

\section{参考文献}

至少列出 10 篇参考文献,且只列出作者直接阅读过或在正文中被引用过的文献资料。 在正文中按出现的先后次序用上标方式在引用位置处标明序号并加方括号。参考文献要另起 一页,一律放在正文之后,不得放在各章节之后。

“参考文献”四个字样式用小三、黑体、加粗、顶格,单倍行距,段前 6 磅,段后 6 磅。各参考文献序号用中扩号,与文字之间空一格。中文字号和字体用小四、宋体,英文字 号和字体用小四、Times New Roman。行距固定值 20 磅,左对齐,悬挂缩进 2 字符。参考文 献作者只写到第三位,余者写“等”,多个作者之间用逗号隔开。

几种主要参考文献的格式为: 专(译)著:作者.书名(译者)[M].出版地:出版者,出版年,起止页码. 期刊:作者.文题[J].刊名,年, 卷号(期号): 起止页码. 论文集:作者.文题[C].编者.文集名.出版地:出版者,出版年,起止页码. 学位论文:作者.文题[博士(或硕士)学位论文][D].授予单位,授予年. 互联网资料:作者.文章标题[EB/OL].完整网址,发表或更新日期/引用日期. 用“[J]”等在文章标题后标识各参考文献的类型。专著“[M]”,论文集“[C]”,期刊

文章“[J]”,学位论文“[D]”;网上资料,如引用的是数据库,用“[DB/OL]”表示;如引 用的是电子文献,用“[EB/OL]”表示。如直接引用网页,用“[Z]”表示。
\section{附录}
另起一页。附录的有无根据毕业论文(设计)的情况而定。

\section{致谢}
 
 谢辞应以简短的文字对在课题研究和论文撰写过程中曾直接给予帮助的人员(例如指导 教师、答疑教师及其他人员)表示自己的谢意,这不仅是一种礼貌,也是对他人劳动的尊重, 是治学者应有的思想作风。
  “致谢”二字样式用小三、黑体、加粗、居中,单倍行距,段前 6 磅,段后 12 磅。内 容限 1 页,采用小四、 楷体\_GB2312 ,行距固定值 20 磅。致谢要另起一页。
  
\section{ 成绩评定表 }
主要包括指导教师对毕业论文(设计)的评语、答辩组综合评定意见以及学院意见等。
\section{答辩记录}
主要包括答辩内容摘要、答辩小组意见等。
\section{论文相似性鉴定报告}
所有毕业论文必须提供中国知网(CNKI)对论文正文部分的相似性检测报告,并在报告 首页由学生本人亲自用黑色水笔手写签名。原则上,纯理论的毕业论文复制比不大于 30\%, 有设计的毕业论文复制比不大于 20\%;若有设计的毕业论文复制比介于 20\%-30\%之间,由指 导教师审核,决定是否通过或需要修改后重新检测。论文中只装订相似性鉴定报告首页。

