% !TeX root = ../Main/HBNUthesis.tex


\hbnuabstract{\hbnutitle} 

%----------------------------------副标题---------------------------------------%


%\begin{flushright}{		 
\hbnusubabstract{\hbnusubtitle }%%右对齐缩进四个字符
%		}
%\end{flushright}

{
\noindent \makebox[4em][s]{\xiaosi\textbf {摘要:}}{摘要尽量写成报道性摘要,其内容独立于正文,要能准确、具体、完整地概括原文的创新之处。中文要不少于200字,英文摘要应与中文摘要相对应。摘要应回答好以下三个方面问题:1、What you want to do(直接写出研究目的,可缺省);2、How you did it(详细陈述过程和方法);3、What results did you get and what conclusions can you draw(全面罗列结果和结论)。中英文摘要一律采用第三人称表述,不使用“本文”、“作者”等作为主语。关键词选词要规范。应尽量从汉语主题词表中选取,未被词表收录的词如果确有必要也可作为关键词选用。中英文关键词应一一对应。}\\
\xiaosi \textbf{关键词:} 关键词1;关键词2;关键词3	}

%---------------------------------英文摘要--------------------------------------%
\newpage
\Ehbnuabstract{\Ehbnutitle}
%----------------------------------副标题---------------------------------------%
%\begin{flushright}{
		\Ehbnusubabstract{\Ehbnusubtitle}%%右对齐缩进四个字符
%		}
%\end{flushright}

{
\noindent \xiaosi \textbf{Abstract:} \setlength{\baselineskip}{20pt}{As far as possible, it should be written as a report summary, whose content is independent of the text, and it should be able to accurately, concretely and completely summarize the innovations of the original text. Chinese abstracts should be no less than 200 words. English abstracts should correspond to Chinese abstracts. The following three questions should be answered well: 1. What do you want to do (write the research purpose directly, but default); 2. How did you do it (state the process and method in detail); 3.What results did you get and what conclusions can you draw. Both Chinese and English abstracts are expressed in the third person without the use of "the text" or "the author" as the subject.}\\
\xiaosi \textbf{Key words:} {keyword1;keyword2;keyword3}}
