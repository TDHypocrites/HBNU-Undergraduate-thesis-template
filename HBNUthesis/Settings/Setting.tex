% !TeX root = ../Main/HBNUthesis.tex
\usepackage{ctex}
%\ctexset{today=big,}

\usepackage{amsmath,amsthm}
\usepackage{amsfonts}
\usepackage{verbatim}
\usepackage{ifthen}
\usepackage{xcolor}
\usepackage{multirow}
\usepackage{longtable}
\usepackage[para]{threeparttable}
\usepackage{array,booktabs,longtable,tabularx}
\usepackage[font={small,bf}]{caption}
\usepackage{setspace}
\usepackage{titletoc}
\usepackage{array}
\usepackage[raggedright]{titlesec}

\usepackage[a4paper,
%bindingoffset=1cm,
left=2.5cm,
right=2.5cm,
top=2.5cm,
bottom=2.5cm,
footskip=1.5cm,
twoside,
]{geometry}

\usepackage{graphicx}
\graphicspath{{../Figures/}}%设定图片的存放路径
\usepackage{tikz}
\newcommand*\circled[1]{\tikz[baseline=(char.base)]{
		\node[shape=circle,draw,inner sep=1pt] (char) {#1};}}
\usepackage{pifont}

%设置代码的插入
\usepackage{listings}
\def\lstbasicfont{\fontfamily{pcr}\selectfont\footnotesize}
\lstset{%
	numbers=left,
	numberstyle=\tiny,
	basicstyle=\small,
	backgroundcolor=\color{white},      % choose the background color
	basicstyle=\footnotesize\ttfamily,  % size of fonts used for the code
	columns=fullflexible,
	tabsize=4,
	breaklines=true,               % automatic line breaking only at whitespace
	captionpos=b,                  % sets the caption-position to bottom
	commentstyle=\color{green},  % comment style
	escapeinside={\%*}{*)},        % if you want to add LaTeX within your code
	keywordstyle=\color{blue},     % keyword style
	stringstyle=\color{purple}\ttfamily,  % string literal style
	frame=single,
	rulesepcolor=\color{red!20!green!20!blue!20},
}
\lstloadlanguages{C,C++,Java,Matlab,Mathematica}

%算法的插入宏包
\usepackage{algorithm}
\usepackage[noend]{algpseudocode}

\makeatletter
\def\BState{\State\hskip-\ALG@thistlm}
\makeatother


%font
\usepackage{fontspec} 
\usepackage{xeCJK}%中文字体
%Windows请保留以下三行代码。而Mac用户请注释以下三行代码,开启后面三行代码。Linux用户请自行配置字体。
%\setCJKmainfont[ItalicFont={楷体}, BoldFont={黑体}]{宋体}%衬线字体 缺省中文字体为
%\setCJKsansfont{黑体}
%\setCJKmonofont{楷体}%中文等宽字体

\setCJKmainfont[ItalicFont={Kai}, BoldFont={Hei}]{STSong}%衬线字体 缺省中文字体为
\setCJKsansfont{Hei}
\setCJKmonofont{Kai}%中文等宽字体

\setmainfont{Times New Roman} %西文部分默认使用的字体
\setsansfont{Times New Roman} %西文默认无衬线字体
\setmonofont{Times New Roman}%西文默认的等宽字体

\usepackage[super]{gbt7714}
\usepackage[sort&compress]{natbib}
\renewcommand{\citep}[1]{{\color{blue}\citeauthor{#1}(\citeyearpar{#1})}}


\setCJKfamilyfont{zhsong}[AutoFakeBold = {2.17}]{SimSun}%%实现宋体加粗
\renewcommand*{\songti}{\CJKfamily{zhsong}}


\usepackage[bookmarks=true,bookmarksnumbered=ture,
colorlinks,linkcolor=black,
citecolor=blue,urlcolor=green]{hyperref}
\usepackage{cleveref}
%\let\oldeqref\eqref
%\renewcommand{\eqref}{公式}\oldeqref}

\crefname{figure}{图}{}
\crefname{table}{表}{}
\crefname{equation}{公式}{}
%equation, chapter, section, etc.
%\creflabelformat{htypei}{hformati}
\newtheorem{theory}{定理}[section]
\theoremstyle{definition}\newtheorem{law}{定律}[section]
\def\lq{`}\def\rq{'}


%%%%%%%%%%%%%%%%%%%%%%%%%%%%%%%%%%%%%%%%%%%%%%%%%%%%%%%%%%%
% 重定义字号命令
%%%%%%%%%%%%%%%%%%%%%%%%%%%%%%%%%%%%%%%%%%%%%%%%%%%%%%%%%%%
\newcommand{\xiaochu}{\fontsize{30pt}{30pt}\selectfont}    % 小初, 单倍行距
\newcommand{\yihao}{\fontsize{26pt}{26pt}\selectfont}    % 一号, 单倍行距
\newcommand{\erhao}{\fontsize{22pt}{22pt}\selectfont}    % 二号, 单倍行距
\newcommand{\xiaoer}{\fontsize{18pt}{18pt}\selectfont}    % 小二, 单倍行距
\newcommand{\sanhao}{\fontsize{16pt}{16pt}\selectfont}    % 三号, 单倍行距
\newcommand{\xiaosan}{\fontsize{15pt}{15pt}\selectfont}    % 小三, 单倍行距
\newcommand{\sihao}{\fontsize{14pt}{14pt}\selectfont}    % 四号, 单倍行距
\newcommand{\banxiaosi}{\fontsize{13pt}{13pt}\selectfont}    % 半小四, 单倍行距
\newcommand{\xiaosi}{\fontsize{12pt}{12pt}\selectfont}    % 小四, 单倍行距
\newcommand{\dawuhao}{\fontsize{11pt}{11pt}\selectfont}    % 大五号, 单倍行距
\newcommand{\wuhao}{\fontsize{10.5pt}{10.5pt}\selectfont}    % 五号, 单倍行距
\newcommand{\xiaowu}{\fontsize{9pt}{9pt}\selectfont}    % 小五号, 单倍行距




 
 
\makeatletter
\setlength{\parskip}{6pt}

\newcommand\hbnuabstract{\@startsection{subsubsection}{3}%
 {0pt}{12pt}{6pt}%
 {\centering\erhao\bfseries\heiti }}
 
 \newcommand\Ehbnuabstract{\@startsection{subsubsection}{3}%
 {0pt}{12pt}{6pt}%
 {\centering\erhao\bfseries\fontspec{Arial} }}
 
 \newcommand\hbnusubabstract{\@startsection{subsubsection}{3}%
 {0pt}{0pt}{24pt} 
 {\songti \bfseries	\sihao\rightline }}
  \newcommand\Ehbnusubabstract{\@startsection{subsubsection}{3}%
 {0pt}{0pt}{24pt} 
 { \fontspec{Arial} \sihao \bfseries\rightline }}
 
  \newcommand\hbnuthanks{\@startsection{chapter}{3}%
 {0pt}{6pt}{12pt} 
 { \bf\xiaosan\center }}
% 设置标题段前段后距离

\titlespacing*{\chapter} {0pt}{6pt}{6pt}
\titlespacing*{\section} {0pt}{6pt}{6pt}
\titlespacing*{\subsection} {0pt}{6pt}{6pt}


\titleformat{\chapter}{\xiaosan\bf}{\thechapter\ }{1em}{}

\renewcommand*\l@chapter[2]{\@dottedtocline{1}{5em}{5em}%
	{\sffamily \songti \bf
		\vspace{0.4em}\hskip 2em  \xiaosan #1}{#2}}

\renewcommand*\l@section[2]{\@dottedtocline{1}{1.5em}{2.5em}%
	{\sffamily \songti 
		\vspace{0.4em}\hskip 2em  \xiaosan #1}{#2}}
\renewcommand*\l@subsection[2]{\@dottedtocline{2}{3.8em}{3em}%
	{\rmfamily  \songti  
		\vspace{0.4em}\hskip 4.5em \sihao #1}{#2}}

% for section
\renewcommand\section{\@startsection{section}{1}{\z@}%
	{6pt\@plus 0pt \@minus 6pt}%
	{6pt\@plus 0pt \@minus 6pt}%
	{\sffamily
		\sihao \bfseries\heiti}}
\renewcommand\subsection{\@startsection{subsection}{2}{\z@}%
	{6pt\@plus 0pt \@minus 6pt}%
	{6pt\@plus 0pt \@minus 6pt}%
	{\sffamily
		\xiaosi \bfseries\heiti}}
%
\bibliographystyle{unsrt}
%
%\renewcommand\thechapter{\arabic{chapter}} 
%\titlecontents{chapter}[0pt]{\songti\vspace{0.4em} \bfseries\sanhao }{}{}{\titlerule*[0.5pc]{.}\contentspage}


%%用于产生没有编号但在目录中列出的章。
%% \phantomsection is the anchor hyperref needed to make a bookmark,
%% without it, hyerref would throw out warnings.
%% typeset Chinese Chapter, then list it in toc and eoc

%% Chinese Chapter only in toc




%%===========================目录==============================%%

%%设置目录格式。
\renewcommand\tableofcontents{%
	\thispagestyle{plain}
	\parskip 0pt
	%    \Cchapter{\texorpdfstring{\contentsname}{目录}}%
	\begin{center}
	\setlength{\parskip}{12pt}
		\heiti{\xiaosan{\textbf{目\qquad 录}}}
		\setlength{\parskip}{12pt}
	\end{center}
	\@starttoc{toc}% 
}

%%============================关键词===========================%%

%%中文关键词。
\newcommand\keywords[1]{%
	\vspace{2ex}\noindent{\sffamily \bfseries 关键词:} #1}


\makeatother
