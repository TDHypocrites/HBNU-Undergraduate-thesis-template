% !TeX root = ../Main/HBNUthesis.tex
\chapter{防火墙技术}
防火墙是计算机网络持续健康发展的重大问题,也是计算机互联网安全保护系统的基本保证,是计算机网络主动防范恶意攻击的有效途径,防火墙技术的持续发展与更新伴随着木马病毒的发展,此消彼长\cite{4} 。


\section{状态监测防火墙}
状态监测防火墙主要是对网络平台在线运行时的状态进行监控和分析,经过一系列特定的数据分析和正则匹配对平台系统的运行状态进行检测,同时对不安全的互联网状态进行相应的处理,进行主动的检测和防御,该防火墙实现对于网络平台的安全保护作用。

因为是对平台系统的整体状态进行检测,所以对响应时间有一定的要求,在应急处理前和处理后会有一定的时间上的缓冲区,而该防火墙相对于其他类型防火墙安全保护等级较高,并且可以根据环境的不同修改需求和一定程度上的规则的拓展和伸缩。

\section{包过滤防火墙}

包过滤防火墙是一种最普通的适用于简单网络的防火墙。在 Internet 网关处使用这种方法只需简单地安装一个数据包过滤路由器, 并设置用户自定义的过滤规则来过滤掉符合相应规则的包。 包过滤防火墙工作在网络层和传输层。 在发送前,针对数据包的过滤器先检查每一个数据包, 根据数据包的 IP 源地址、IP 目的地址、所用的 TCP 源端口号和目的端口号、TCP 链路状态等因素或它们的组合来确定是否允许数据包通过, 只有满足过滤逻辑的数据包才能被转发至相应的目的地的输出端口, 其余数据包则从数据流中被删除。

\section{应用型防火墙}

应用型防火墙主要是通过对IP地址或者端口的伪装,冒充或者诱导有害的入侵行为。该防火墙通过转换IP地址的方式在入侵者与被保护主机之间形成一道无形的隔离层,达到真正的通信流量阻隔的作用,该防火墙类型对硬件处理速度和匹配请求算法有一定的要求,与此同时会使网络变得更为复杂,同时提高了网络人工维护的成本。




%\begin{algorithm}  
%        \caption{用归并排序求逆序数}  
%        \begin{algorithmic}[1] %每行显示行号  
%            \Require $Array$数组,$n$数组大小  
%            \Ensure 逆序数  
%            \Function {MergerSort}{$Array, left, right$}  
%            \EndFunction  
%        \end{algorithmic}  
%    \end{algorithm}  





