% !TeX root = ../Main/HBNUthesis.tex
\chapter{特洛伊木马的现状}
\section{木马的定义}
传统意义上的后门往往就是能够让攻击者获得一个操控当前主机的一个终端,也就是shell,通过这个shell进行一些操作,
而木马是在宿主计算机上启动,在计算机用户完全不知道的前提下,攻击者可以通过由后门发起的端口请求获得一系列操作的权限,进而能够获得远程访问和系统访问的权限。

目前主流的木马可以分为端口监听型、端口反射型两种连接方式,现在常见的木马有网游木马、广告木马等。



\section{木马的特点}
虽然在互联网中存在各种各样的后门脚本,但是从一定的高度来看,这些后门脚本的特征大致可以分为隐蔽性、自动运行性、欺骗性这五大类三种后门木马脚本。
\begin{itemize}
\item{隐蔽性}


隐蔽性是指木马需要隐藏在宿主主机中,避免被主机用户发现的一种性质。这是因为设计木马的作者往往都不会让宿主用户发现自己的木马。木马脚本的隐蔽性主要有两个方面的体现\cite{7} :第一是不会在宿主主机的明显的位置产生容易发现的图标,第二是木马脚本会在任务管理器等显示任务或者进程的窗口中隐藏,与此同时能够以系统服务的状态运行,以欺骗宿主主机操作系统。

\item{自用运行性}


木马的自动运行性是指木马脚本会随着操作系统的调起同时调用,作为伪装的系统服务或者机器运行必要程序,所以木马脚本必须进入计算机操作系统的自启动配置文件中,如启动组或系统进程。

\item{欺骗性}


木马之所以具有欺骗性是需要对目标用户具有一定的迷惑性,使其不能在短时间内低成本完成病毒的发现和清理工作。为此,木马常常是使用系统内常见的服务或者软件的名称,使用这种不容易区分的方式迷惑用户。
\end{itemize}

\section{穿墙木马技术手段}
具备可穿墙的木马的优点除了上述木马所具备的通用特点外,可穿墙漏洞可以实现在宿主主机完全不知情的情况下冒充正常主机的访问和使用来达到自己向远程主机传输命令的任务。特洛伊木马可以采用各种不同的方法实现自己获得与攻击主机连接的目标,目前主流的防火墙防范方式有以下几种。
\begin{itemize}
\item{端口筛选}

通过筛选 TCP/IP 端口,控制到达服务器主机或其他各类网络设备的通信类型。虽然在 Internet 访问点部署的防火墙通常用于限制流入专用网络的流量,但是网络防火墙可能无法保护服务器不被“后门”攻击或内部攻击,这些攻击源于专用网络内的恶意用户\cite{2}。

\item{应用程序筛选}

在应用程序筛选的情况下,只有特定的应用程序可以访问网络\cite{3}。由此进程的插入技术诞生了,通常防火墙会默认允许一些常用的应用程序通过访问网络,于是木马便盯上了这些程序。	

现在的主流的远程控制通常都是插入进程式,一是隐蔽,二是穿墙.典型如替换系统服务BITS,插入svchost.exe中的Bits.dll和默认插入iexplorer.exe浏览器进程的灰鸽子/PcShare等.
 
\item{敏感信息检测}

现在很多防火墙都可以检查诸如用户口令之类的敏感信息,所以抗IDS,抗自动分析,这就成了最新的高级木马必须要注意的地方也就是说需要保证黑客在作出隐蔽操作时需要保证其自身的安全和隐蔽.加密措施是相对而已比较常用也比较简单的方法,还有如最简单的对付IDS检测的方法,xor异或加密\cite{5} 。而现在防火墙自然不是其中一个独立实现,而是将多种技术相互融合进行集成。
  
\end{itemize}


