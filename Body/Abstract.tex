% !TeX root = ../Main/HBNUthesis.tex


\hbnuabstract{\hbnutitle} 

%----------------------------------副标题---------------------------------------%
 
%\hbnusubabstract{\hbnusubtitle }


{
\noindent \makebox[4em][s]{\xiaosi\textbf {摘要:}}{本文先介绍当前的木马情况,描述基本的防火墙技术基础和一些基础的配置和文章内容和结构。论文着重描述在主流防火墙的的情况下特洛伊后门木马如何在宿主主机用户毫无预兆的与攻击机主机建立TCP连接互相通信,进而实现宿主主机的代码执行。实现代码主要分成三个部分,主函数负责判断用户输入的参数的合法性以及脚本模式;服务端类模式下的数据通信,参数传递等;客户端模式下的数据通信,数据包处理,代码执行。由此构成完整的后门程序。对比各类木马类型和实现方法后,选择其中一种进行设计,并完成多平台测试。最后对研究过程中出现的问题进行总结,并为进一步研究提出了方向。}\\
\xiaosi \textbf{关键词:} 后门;防火墙;python}

%---------------------------------英文摘要--------------------------------------%
\newpage
\Ehbnuabstract{\Ehbnutitle}
%----------------------------------副标题---------------------------------------%
%\Ehbnusubabstract{\Ehbnusubtitle}

{
\noindent \xiaosi \textbf{Abstract:} \setlength{\baselineskip}{20pt}{This paper first introduces the current situation of Trojan, describe the basic firewall technology foundation and some basic configuration and the content and structure. This paper focuses on the description of the mainstream firewall in the case of the Trojan horse in the parasitifer host user without warning and attack host to establish a TCP connection to communicate with each other, and then achieve the parasitifer host code execution. The implementation code is mainly divided into three parts. The main function is responsible for judging the legitimacy of the parameters input by the user and the script mode. Data communication and parameter transfer under the server-side class mode; Client mode data communication, packet processing, code execution. This constitutes a complete backdoor program. After comparing each kind of Trojan horse type and the realization method, selects one of them to carry on the design, and completes the multi-platform test. Finally, the problems in the research process are summarized, and the direction for further research is proposed.}\\
\xiaosi \textbf{Key words:} {backdoor ; firewall ; python }}
