\chapter{其他要求}
 
\section{文字}

论文中汉字应采用《简化汉字总表》规定的简化字,并严格执行汉字的规范。所有文字 字面清晰,不得涂改。

\section{表格}
论文的表格逐章单独编号(如:如第 2 章第 3 个表编号为:表 2-3),表编号必须连续, 不得重复或跳跃。表编号和表名称置于表格上方,样式用五号、宋体、居中,单倍行距;表 中文字中文样式用五号、宋体,英文样式用五号、Times New Roman;整个表居中对齐。

表格的结构应简洁。表格中各栏都应标注量和相应的单位。表格内数字须上下对齐,相 邻栏内的数值相同时,不能用‘同上’、‘同左’和其它类似用词,应一一重新标注。
\section{图}

论文的图逐章单独编号(如:如第 3 章第 4 个图编号为:图 3-4),图编号必须连续, 不得重复或跳跃。图编号和图名称置于图下方,样式用五号、宋体,单倍行距。整个图居中 对齐。

插图要精选。毕业论文(设计)中的插图以及图中文字符号应打印,无法打印时一律用 钢笔绘制和标出。

由若干个分图组成的插图,分图用 a,b,c,......标出。
\section{公式}

论文中重要的或者后文中须重新提及的公式应注序号并加圆括号,序号一律用阿拉伯数
字逐章编号(如:第 4 章第 2 个公式编号为:(4.2)),序号排在版面右侧,且距右边距 离相等。公式与序号之间不加虚线。
\section{数字用法}

公历世纪、年代、年、月、日、时间和各种计数、计量,均用阿拉伯数字。年份不能简
写,如 2014 年不能写成 14 年。数值的有效数字应全部写出,如:0.60:2.00 不能写作 0.6:2。 6.计量单位的定义和使用方法按国家计量局规定执行。

