\chapter{总结}
项目中的后门程序实现的一般木马程序的基本功能,但是在该木马程序实际运行使用的过程中,我感受到了很多不足,这里将对项目中的工作进行总结,并将部分不足描述出来,并给出部分解决方法的初步讨论。
\section{工作总结}
在后门实现的过程中,自己完成了对代码的整体设计和绕过方式的实现,利用主流防火墙的一些弱点,实现了后门对防火墙的绕过。在项目启动时我准备了很多关于防火墙和最新木马的材料,结合各大科技论坛,加上自己对特洛伊木马的理解,使我对后门程序有了深刻的认识,这些零碎的知识点都对我在后期实现和设计后门程序的过程中起到了极其重要的作用。自己利用python的特性通过socket编程完成在宿主主机上实现对外反弹shell的过程。代码的简洁性和可读性还有一定的提升空间,在研究中越发的意识到自己对于计算机网络和计算机基础知识的不足,当然,整个后门程序的设计,还有很多问题和不足,希望各位老师批评指正。

该后门程序基本完成了shell穿过防火墙的任务,该shell后门可以完成在宿主主机用户没有发觉的情况下实现对宿主主机的控制,包括文件的读取,写入还有基本的命令的执行。
\section{进一步研究方向}
\subsection{隐蔽性}
在宿主主机中,后门程序的升级其中一个主要的方面就是其隐蔽性。要使后门程序能够具有隐蔽性功能,一种方法是使用线程插入的方法,使得在线程控制查询时不会暴露出自己,实现隐蔽;另一种方法是采用与系统相近应用程序的命名;最后可以使用RootKit技术实现基于底层的木马特征的隐藏。

这些方法都能够降低后门程序被宿主主机用户发现的可能性,近一步完善木马后门程序的功能。

\subsection{运行环境要求}
因为该木马程序是由python编写而成,在Linux系统中可以较为自由的运行使用,但是在部分未安装python
环境的Linux系统或者windows系统中该脚本不能很方便的运行,这里就要求有更好的兼容性和跨平台性,这里有简单的两个基本的可实现的方式,一是将python脚本编译成可执行的二进制文件,能够实现跨平台运行使用,另一种方法,因为大多数Linux系统通过终端安装所需要的python环境的可行性较高,我们可以通过将.py脚本通过编译的方法生成exe文件,使其上传到windows宿主主机后可以完好运行。










